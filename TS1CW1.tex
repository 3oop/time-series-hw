\documentclass{article}

\usepackage{amsmath}
\usepackage{xepersian}
\usepackage{fontspec}
\settextfont[Scale=1]{B Mitra}
\setlatintextfont[Scale=1]{Times New Roman}
\usepackage{fullpage}
% font for linux 
% \setlatintextfont[Scale=1]{TeX Gyre Termes}

\title{معادله مشحصه سری اتورگرسیو مرتبه ۳}
\author{پوریا عصاره‌ها}
\date{\today}

\begin{document}
\maketitle

\section{\lr{AR(3)} معادله مشخصه}

\begin{equation}
    Y_t = \phi_1 Y_{t-1} + \phi_2 Y_{t-2} + \phi_3 Y_{t-3} + e_t 
\end{equation} 

$Y_t$ را اینگونه به صورت برداری تعریف می کنیم.
\[
    Y_t = \left[ \begin{matrix}
        Y_{t+2} \\ Y_{t+1} \\ Y_{t}
    \end{matrix}  \right] \qquad , \qquad 
    e_t = \left[ \begin{matrix}
        e_{t+2} \\ 0 \\ 0 
    \end{matrix} \right] 
\]

آنگاه
\[ Y_1 = A Y_0 + e_1 \quad \Rightarrow \quad \left[ \begin{matrix}
    Y_{3} \\ Y_{2} \\ Y_{1}
\end{matrix} \right]  = A \left[ \begin{matrix}
    Y_{2} \\ Y_{1} \\ Y_{0}
\end{matrix} \right] + \left[ \begin{matrix}
    e_{3} \\ 0 \\ 0 
\end{matrix} \right]\]
و
\[ 
    Y_{t+1} = A^t Y_0 + \sum_i=0^t A^{t-i} e_i  
\]
بر اساس رابطه 1 می‌توان ماتریس \lr{A} را پیدا کرد.

\[ A = \left[ \begin{matrix}
    \phi_1 & \phi_2 & \phi_3 \\
    1      &    0   &   0    \\
    0      &    1   &   0    \\
\end{matrix} \right] \]

از جبر خطی می‌دانیم می‌دانیم 
\[ 
    A^n = U D^n U^{-1}        
\]

که \lr{D} ماتریس قطری مقادیر ویژه و \lr{U} ماتریس بردارهای ویژه است.

برای اینکه شرط مانایی بر قرار باشد، اندازه $A^t$ وقتی $t$ به بی نهایت میل می‌کند باید محدود باشد.

یعنی
\[
    t \rightarrow \infty \; : \; UD^tU^{-1} < \infty \quad \Rightarrow \quad \lambda^t < \infty
\]

$\lambda$ها مقادیر ویژه ماتریس
\lr{A} هستند.

\[ \lim_{t\rightarrow \infty} \lambda^t < \infty \quad \Rightarrow \quad | \lambda | < 0 \]

لامداها با حل این معادله بدست می‌آیند
\begin{equation}
    |A - \lambda I | = 0 
\Rightarrow \Bigg| \begin{matrix}
    \phi_1 -\lambda & \phi_2 & \phi_3 \\
    1      &    -\lambda   &   0    \\
    0      &    1   &   -\lambda    \\
\end{matrix} \Bigg| = -\lambda^3 +\lambda^2 \phi_1 + \lambda \phi_2 + \phi_3 = 0
\end{equation}

با تغییر متغییر $x = \frac{1}{\lambda} $ و $ |x| > 1 $ 
معادله را ساده‌تر کرد. از آنجایی که $x \neq 0 $
$-x^3$ را در رابطه 2 ضرب می‌کنیم.

\[
    x^3\lambda^3 -x^3\lambda^2 \phi_1 - x^3\lambda \phi_2 - x^3\phi_3 = 0 \]
\begin{equation}
    1 - x \phi_1 - x^2 \phi_2 - \phi_3 x^3 = 0
\end{equation}

حالا شرط مانایی این خواهد بود که قدرمطلق ریشه‌های $x$ 
بزرگ تر از 1 باشند.

\section{تابع خودکواریانس}

\[ \gamma_k = Cov(Y_t, Y_{t-k}) = Cov(\phi_1 Y_{t-1} + \phi_2 Y_{t-2} + \phi_3 Y_{t-3} + e_t , Y_{t-k})\]
\begin{equation}
    \gamma_k = \phi_1 Cov(Y_{t-1}, Y_{t-k}) + \phi_2 Cov(Y_{t-2}, Y_{t-k}) + \phi_3 Cov(Y_{t-3}, Y_{t-k}) + Cov(e_t , Y_{t-k})
\end{equation}

$k = 0$
\[ \gamma_0 = \phi_1 \gamma_1 + \phi_2 \gamma_2 + \phi_3 \gamma_3 \] 
یا به صورت مستقیم
\begin{align*}
    & \gamma_0 = Cov(Y_t, Y_t) = Var(Y_t) \\ 
    = & Var(\phi_1 Y_{t-1} + \phi_2 Y_{t-2} + \phi_3 Y_{t-3} + e_t) \\
    = & \sigma^2 + \phi_1^2 Var(Y_{t-1}) + \phi_2^2 Var(Y_{t-2}) + \phi_3^2 Var(Y_{t-3}) \\
    + & \phi_1\phi_2Cov(Y_{t-1}, Y_{t-2}) + \phi_2\phi_3Cov(Y_{t-2}, Y_{t-3}) + \phi_1\phi_3Cov(Y_{t-1}, Y_{t-3}) \\ 
    = & \sigma^2 + \phi_1^2\gamma_0 + \phi_2^2\gamma_0 + \phi_3^2\gamma_0 + \phi_1\phi_2\gamma_1 + \phi_2\phi_3\gamma_1 + \phi_1\phi_3\gamma_2 \\
    \Rightarrow & \gamma_0 = \frac{\sigma^2 + \phi_1\phi_2\gamma_1 + \phi_2\phi_3\gamma_1 + \phi_1\phi_3\gamma_2}{1 - \phi_1^2 - \phi_2^2 - \phi_3^2 }
\end{align*}


\[ \gamma_k = \phi_1\gamma_{k-1} + \phi_2\gamma_{k-2} + \phi_3\gamma_{k-3} \]

% $k = 1$

% \begin{align*}
%     & \gamma_1 = \phi_1 Cov(Y_{t-1}, Y_{t-1}) + \phi_2 Cov(Y_{t-2}, Y_{t-1}) + \phi_3 Cov(Y_{t-3}, Y_{t-1}) + Cov(e_t , Y_{t-1}) \\
%     & = \phi_1\gamma_0 + \phi_2\gamma_1 + \phi_3\gamma_2 + 0 \quad (*)\\ 
%     & \gamma_2 =  Cov(Y_{t-3}, Y_{t-1}) = \phi_1 Cov(Y_{t-3}, Y_{t-2}) + \phi_2 Cov(Y_{t-3}, Y_{t-3}) + \phi_3 Cov(Y_{t-3}, Y_{t-4}) + Cov(Y_{t-3} , e_{t-1} ) \\
%     & \gamma_2 = \phi_1\gamma_1 + \phi_2\gamma_0 + \phi_3\gamma_1 + 0 \quad (**) \\
%     & (*) , (**) \Rightarrow \quad \gamma_1 = \phi_1\gamma_0 + \phi_2\gamma_1 + \phi_3 \left( \phi_1\gamma_1 + \phi_2\gamma_0 + \phi_3\gamma_1  \right) \\
%     & \gamma_1 (1 - \phi_2) = \phi_1\gamma_0 + \phi_3 \phi_2 \gamma_0 + \gamma_1 \phi_3^2 + \gamma_1 \phi_3\phi_1  \\
%     & \gamma_1 = \frac{\left( \phi_1+ \phi_3 \phi_2 \right) \left( \frac{\sigma^2}{1 - \phi_1^2 - \phi_2^2 - \phi_3^2 }\right)}{1 - \phi_2 -  \phi_3 (\phi_1 + \phi_3)}
% \end{align*}

% $k = 2$ بر اساس $(**)$ و $(*)$

% \begin{align*}
%     \gamma_2 = \phi_1 \frac{\left( \phi_1+ \phi_3 \phi_2 \right) \left( \frac{\sigma^2}{1 - \phi_1^2 - \phi_2^2 - \phi_3^2 }\right)}{1 - \phi_2 -  \phi_3 (\phi_1 + \phi_3)}
%     + \phi_2 \frac{\sigma^2}{1 - \phi_1^2 - \phi_2^2 - \phi_3^2 }
%     + \phi_3\frac{\left( \phi_1+ \phi_3 \phi_2 \right) \left( \frac{\sigma^2}{1 - \phi_1^2 - \phi_2^2 - \phi_3^2 }\right)}{1 - \phi_2 -  \phi_3 (\phi_1 + \phi_3)}
% \end{align*}

\section{معادلات یول واکر}

\[
\begin{cases}
    \rho_1 = \phi_1 + \phi_2\rho_2 + \phi_3 \rho_3 \\
    \rho_2 = \phi_1\rho_1 + \phi_2 + \phi_3 \rho_3 \\
    \rho_3 = \phi_1\rho_1 + \phi_2 + \phi_3 
\end{cases} \Rightarrow \begin{cases}
    - \rho_1 + \phi_1 + \phi_2\rho_2 + \phi_3 \rho_3 = 0 \\
    - \rho_2 + \phi_1\rho_1 + \phi_2 + \phi_3 \rho_3 = 0 \\
    - \rho_3 + \phi_1\rho_1 + \phi_2 + \phi_3 = 0 
\end{cases}
\]

\[ \left[\begin{matrix}
    -1 & \phi_2 & \phi_3 \\
    \phi_1 & -1 & \phi_3 \\
    \phi_1 & \phi_2 & -1
\end{matrix}\right] \left[ \begin{matrix}
    \rho_1 \\ \rho_2 \\ \rho_3
\end{matrix} \right] + \left[ \begin{matrix}
    \phi_1 \\ \phi_2 \\ \phi_3
\end{matrix} \right] = 0
 \]

با یافتن وارون ماتریس ضرایب معادله را حل می‌کنیم.

\[ \left[\begin{matrix}
    -1 & \phi_2 & \phi_3 \\
    \phi_1 & -1 & \phi_3 \\
    \phi_1 & \phi_2 & -1
\end{matrix}\right]^{-1} = \frac{1}{\left|\begin{matrix}
    -1 & \phi_2 & \phi_3 \\
    \phi_1 & -1 & \phi_3 \\
    \phi_1 & \phi_2 & -1
\end{matrix}\right|}Adj\left(\begin{matrix}
    -1 & \phi_2 & \phi_3 \\
    \phi_1 & -1 & \phi_3 \\
    \phi_1 & \phi_2 & -1
\end{matrix}\right) \]
\[= \frac{1}{\phi_2\phi_3 - 1 + \phi_2\phi_1(1+\phi_3) + \phi_3\phi_1(1+\phi_2)}\left[\begin{matrix}
    1 - \phi_2\phi_3    &   \phi_1(1+\phi_3)    & \phi_1(1+\phi_2)\\
    \phi_2(1+\phi_3)    &   1-\phi_1\phi-3      & \phi_2(1+\phi_1) \\
    \phi_3(1+\phi_2)    &   \phi_3(1+\phi_1)    &   1-\phi_1\phi_2\\
\end{matrix}\right]\]

\[ \left[ \begin{matrix}
    \rho_1 \\ \rho_2 \\ \rho_3
\end{matrix} \right] = - \frac{1}{\phi_2\phi_3 - 1 + \phi_2\phi_1(1+\phi_3) + \phi_3\phi_1(1+\phi_2)}
\left[\begin{matrix}
    1 - \phi_2\phi_3    &   \phi_1(1+\phi_3)    & \phi_1(1+\phi_2)\\
    \phi_2(1+\phi_3)    &   1-\phi_1\phi-3      & \phi_2(1+\phi_1) \\
    \phi_3(1+\phi_2)    &   \phi_3(1+\phi_1)    &   1-\phi_1\phi_2\\
\end{matrix}\right] \left[ \begin{matrix}
    \phi_1 \\ \phi_2 \\ \phi_3
\end{matrix} \right]\]

\[ \left[ \begin{matrix}
    \rho_1 \\ \rho_2 \\ \rho_3
\end{matrix} \right] = \left[\begin{matrix}
    \frac{\phi_1(1 - \phi_2\phi_3)  + \phi_2\phi_1(1+\phi_3) + \phi_3\phi_1(1+\phi_2)}{\phi_2\phi_3 - 1 + \phi_2\phi_1(1+\phi_3) + \phi_3\phi_1(1+\phi_2)} \\
    \frac{\phi_1\phi_2(1+\phi_3) + \phi_2(1-\phi_1\phi_3) + \phi_3\phi_2(1+\phi_1)}{\phi_2\phi_3 - 1 + \phi_2\phi_1(1+\phi_3) + \phi_3\phi_1(1+\phi_2)}\\
    \frac{\phi_1\phi_3(1+\phi_2) +\phi_2\phi_3(1+\phi_1) + \phi_3(1-\phi_1\phi_2)}{\phi_2\phi_3 - 1 + \phi_2\phi_1(1+\phi_3) + \phi_3\phi_1(1+\phi_2)}\\
\end{matrix}\right] = \left[\begin{matrix}
    \frac{\phi_1\left(1 + \phi_2 + \phi_3 + \phi_2\phi_3\right)}{\phi_1\phi_2 + \phi_1\phi_3 + \phi_2\phi_3 + 2\phi_1\phi_2\phi_3 - 1} \\ \\
    \frac{\phi_2\left(1 + \phi_1 + \phi_3 + \phi_1\phi_3\right)}{\phi_1\phi_2 + \phi_1\phi_3 + \phi_2\phi_3 + 2\phi_1\phi_2\phi_3 - 1} \\ \\
    \frac{\phi_3\left(1 + \phi_1 + \phi_2 + \phi_1\phi_2\right)}{\phi_1\phi_2 + \phi_1\phi_3 + \phi_2\phi_3 + 2\phi_1\phi_2\phi_3 - 1}
\end{matrix}\right]
\]
\end{document}