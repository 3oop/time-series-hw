% Options for packages loaded elsewhere
\PassOptionsToPackage{unicode}{hyperref}
\PassOptionsToPackage{hyphens}{url}
%
\documentclass[
]{article}
\usepackage{amsmath,amssymb}
\usepackage{iftex}
\ifPDFTeX
  \usepackage[T1]{fontenc}
  \usepackage[utf8]{inputenc}
  \usepackage{textcomp} % provide euro and other symbols
\else % if luatex or xetex
  \usepackage{unicode-math} % this also loads fontspec
  \defaultfontfeatures{Scale=MatchLowercase}
  \defaultfontfeatures[\rmfamily]{Ligatures=TeX,Scale=1}
\fi
\usepackage{lmodern}
\ifPDFTeX\else
  % xetex/luatex font selection
\fi
% Use upquote if available, for straight quotes in verbatim environments
\IfFileExists{upquote.sty}{\usepackage{upquote}}{}
\IfFileExists{microtype.sty}{% use microtype if available
  \usepackage[]{microtype}
  \UseMicrotypeSet[protrusion]{basicmath} % disable protrusion for tt fonts
}{}
\makeatletter
\@ifundefined{KOMAClassName}{% if non-KOMA class
  \IfFileExists{parskip.sty}{%
    \usepackage{parskip}
  }{% else
    \setlength{\parindent}{0pt}
    \setlength{\parskip}{6pt plus 2pt minus 1pt}}
}{% if KOMA class
  \KOMAoptions{parskip=half}}
\makeatother
\usepackage{xcolor}
\usepackage[margin=1in]{geometry}
\usepackage{color}
\usepackage{fancyvrb}
\newcommand{\VerbBar}{|}
\newcommand{\VERB}{\Verb[commandchars=\\\{\}]}
\DefineVerbatimEnvironment{Highlighting}{Verbatim}{commandchars=\\\{\}}
% Add ',fontsize=\small' for more characters per line
\usepackage{framed}
\definecolor{shadecolor}{RGB}{248,248,248}
\newenvironment{Shaded}{\begin{snugshade}}{\end{snugshade}}
\newcommand{\AlertTok}[1]{\textcolor[rgb]{0.94,0.16,0.16}{#1}}
\newcommand{\AnnotationTok}[1]{\textcolor[rgb]{0.56,0.35,0.01}{\textbf{\textit{#1}}}}
\newcommand{\AttributeTok}[1]{\textcolor[rgb]{0.13,0.29,0.53}{#1}}
\newcommand{\BaseNTok}[1]{\textcolor[rgb]{0.00,0.00,0.81}{#1}}
\newcommand{\BuiltInTok}[1]{#1}
\newcommand{\CharTok}[1]{\textcolor[rgb]{0.31,0.60,0.02}{#1}}
\newcommand{\CommentTok}[1]{\textcolor[rgb]{0.56,0.35,0.01}{\textit{#1}}}
\newcommand{\CommentVarTok}[1]{\textcolor[rgb]{0.56,0.35,0.01}{\textbf{\textit{#1}}}}
\newcommand{\ConstantTok}[1]{\textcolor[rgb]{0.56,0.35,0.01}{#1}}
\newcommand{\ControlFlowTok}[1]{\textcolor[rgb]{0.13,0.29,0.53}{\textbf{#1}}}
\newcommand{\DataTypeTok}[1]{\textcolor[rgb]{0.13,0.29,0.53}{#1}}
\newcommand{\DecValTok}[1]{\textcolor[rgb]{0.00,0.00,0.81}{#1}}
\newcommand{\DocumentationTok}[1]{\textcolor[rgb]{0.56,0.35,0.01}{\textbf{\textit{#1}}}}
\newcommand{\ErrorTok}[1]{\textcolor[rgb]{0.64,0.00,0.00}{\textbf{#1}}}
\newcommand{\ExtensionTok}[1]{#1}
\newcommand{\FloatTok}[1]{\textcolor[rgb]{0.00,0.00,0.81}{#1}}
\newcommand{\FunctionTok}[1]{\textcolor[rgb]{0.13,0.29,0.53}{\textbf{#1}}}
\newcommand{\ImportTok}[1]{#1}
\newcommand{\InformationTok}[1]{\textcolor[rgb]{0.56,0.35,0.01}{\textbf{\textit{#1}}}}
\newcommand{\KeywordTok}[1]{\textcolor[rgb]{0.13,0.29,0.53}{\textbf{#1}}}
\newcommand{\NormalTok}[1]{#1}
\newcommand{\OperatorTok}[1]{\textcolor[rgb]{0.81,0.36,0.00}{\textbf{#1}}}
\newcommand{\OtherTok}[1]{\textcolor[rgb]{0.56,0.35,0.01}{#1}}
\newcommand{\PreprocessorTok}[1]{\textcolor[rgb]{0.56,0.35,0.01}{\textit{#1}}}
\newcommand{\RegionMarkerTok}[1]{#1}
\newcommand{\SpecialCharTok}[1]{\textcolor[rgb]{0.81,0.36,0.00}{\textbf{#1}}}
\newcommand{\SpecialStringTok}[1]{\textcolor[rgb]{0.31,0.60,0.02}{#1}}
\newcommand{\StringTok}[1]{\textcolor[rgb]{0.31,0.60,0.02}{#1}}
\newcommand{\VariableTok}[1]{\textcolor[rgb]{0.00,0.00,0.00}{#1}}
\newcommand{\VerbatimStringTok}[1]{\textcolor[rgb]{0.31,0.60,0.02}{#1}}
\newcommand{\WarningTok}[1]{\textcolor[rgb]{0.56,0.35,0.01}{\textbf{\textit{#1}}}}
\usepackage{graphicx}
\makeatletter
\def\maxwidth{\ifdim\Gin@nat@width>\linewidth\linewidth\else\Gin@nat@width\fi}
\def\maxheight{\ifdim\Gin@nat@height>\textheight\textheight\else\Gin@nat@height\fi}
\makeatother
% Scale images if necessary, so that they will not overflow the page
% margins by default, and it is still possible to overwrite the defaults
% using explicit options in \includegraphics[width, height, ...]{}
\setkeys{Gin}{width=\maxwidth,height=\maxheight,keepaspectratio}
% Set default figure placement to htbp
\makeatletter
\def\fps@figure{htbp}
\makeatother
\setlength{\emergencystretch}{3em} % prevent overfull lines
\providecommand{\tightlist}{%
  \setlength{\itemsep}{0pt}\setlength{\parskip}{0pt}}
\setcounter{secnumdepth}{-\maxdimen} % remove section numbering
\ifLuaTeX
  \usepackage{selnolig}  % disable illegal ligatures
\fi
\IfFileExists{bookmark.sty}{\usepackage{bookmark}}{\usepackage{hyperref}}
\IfFileExists{xurl.sty}{\usepackage{xurl}}{} % add URL line breaks if available
\urlstyle{same}
\hypersetup{
  pdftitle={TSHW3},
  pdfauthor={Pooria Assarehha},
  hidelinks,
  pdfcreator={LaTeX via pandoc}}

\title{TSHW3}
\author{Pooria Assarehha}
\date{2023-11-08}

\begin{document}
\maketitle

\hypertarget{time-series-i-hw3}{%
\section{Time Series I HW3}\label{time-series-i-hw3}}

\hypertarget{pooria-assarehha}{%
\subsubsection{Pooria Assarehha}\label{pooria-assarehha}}

\hypertarget{ex-3.3}{%
\subsection{Ex 3.3}\label{ex-3.3}}

if \(Y_t = \mu + e_t + e_{t-1}\) \[
Var(\bar{Y}) = \frac{1}{n^2}Var(\sum^n_i Y_i) = \frac{1}{n^2}Var(\sum_i^n (\mu + e_i + e_{i-1})) = \frac{1}{n^2} Var(e_0 + e_n + 2\sum_i^{n-1} e_i ) = \frac{\sigma^2 + \sigma^2 + 4(n-1)\sigma^2}{n^2} = \frac{(4n-2)\sigma^2}{n^2}
\] if \(Y_t = \mu + e_t\)

\[
Var(\bar{Y}) = \frac{1}{n^2}Var(\sum_i^n \mu + e_i ) = \frac{\sigma^2}{n^2}
\]

\hypertarget{ex-3.5}{%
\subsection{Ex 3.5}\label{ex-3.5}}

\begin{Shaded}
\begin{Highlighting}[]
\FunctionTok{library}\NormalTok{(TSA)}
\end{Highlighting}
\end{Shaded}

\begin{verbatim}
## 
## Attaching package: 'TSA'
\end{verbatim}

\begin{verbatim}
## The following objects are masked from 'package:stats':
## 
##     acf, arima
\end{verbatim}

\begin{verbatim}
## The following object is masked from 'package:utils':
## 
##     tar
\end{verbatim}

\begin{Shaded}
\begin{Highlighting}[]
\FunctionTok{data}\NormalTok{(wages)}
\NormalTok{time.wages }\OtherTok{=} \FunctionTok{time}\NormalTok{(wages)}
\FunctionTok{plot}\NormalTok{(wages)}
\end{Highlighting}
\end{Shaded}

\includegraphics{TS1HW3_files/figure-latex/unnamed-chunk-1-1.pdf}

\begin{Shaded}
\begin{Highlighting}[]
\NormalTok{linfit.wages }\OtherTok{=} \FunctionTok{lm}\NormalTok{(wages}\SpecialCharTok{\textasciitilde{}}\FunctionTok{time}\NormalTok{(wages))}
\FunctionTok{plot}\NormalTok{(wages)}
\FunctionTok{abline}\NormalTok{(linfit.wages, }\AttributeTok{col=}\StringTok{\textquotesingle{}red\textquotesingle{}}\NormalTok{)}
\end{Highlighting}
\end{Shaded}

\includegraphics{TS1HW3_files/figure-latex/unnamed-chunk-2-1.pdf}

\begin{Shaded}
\begin{Highlighting}[]
\NormalTok{res }\OtherTok{=}\NormalTok{ linfit.wages}\SpecialCharTok{$}\NormalTok{residuals}
\NormalTok{std.res }\OtherTok{=}\NormalTok{ res }\SpecialCharTok{{-}} \FunctionTok{mean}\NormalTok{(res)}
\NormalTok{std.res }\OtherTok{=}\NormalTok{ std.res}\SpecialCharTok{/}\FunctionTok{sd}\NormalTok{(res)}
\FunctionTok{plot}\NormalTok{(std.res, }\AttributeTok{type =} \StringTok{\textquotesingle{}o\textquotesingle{}}\NormalTok{)}
\end{Highlighting}
\end{Shaded}

\includegraphics{TS1HW3_files/figure-latex/unnamed-chunk-3-1.pdf}

\begin{Shaded}
\begin{Highlighting}[]
\NormalTok{quad.fit }\OtherTok{=} \FunctionTok{lm}\NormalTok{(wages}\SpecialCharTok{\textasciitilde{}}\NormalTok{time.wages}\SpecialCharTok{+}\FunctionTok{I}\NormalTok{(time.wages}\SpecialCharTok{\^{}}\DecValTok{2}\NormalTok{))}
\FunctionTok{plot}\NormalTok{(wages)}
\FunctionTok{lines}\NormalTok{(}\AttributeTok{x=}\FunctionTok{as.vector}\NormalTok{(time.wages), }\AttributeTok{y=}\FunctionTok{fitted}\NormalTok{(quad.fit), }\AttributeTok{col=}\StringTok{\textquotesingle{}red\textquotesingle{}}\NormalTok{)}
\end{Highlighting}
\end{Shaded}

\includegraphics{TS1HW3_files/figure-latex/unnamed-chunk-4-1.pdf}

\begin{Shaded}
\begin{Highlighting}[]
\NormalTok{qres }\OtherTok{=}\NormalTok{ quad.fit}\SpecialCharTok{$}\NormalTok{residuals}
\NormalTok{std.qres }\OtherTok{=}\NormalTok{ qres }\SpecialCharTok{{-}} \FunctionTok{mean}\NormalTok{(qres)}
\NormalTok{std.qres }\OtherTok{=}\NormalTok{ std.qres}\SpecialCharTok{/}\FunctionTok{sd}\NormalTok{(qres)}
\FunctionTok{plot}\NormalTok{(std.qres, }\AttributeTok{type=}\StringTok{\textquotesingle{}o\textquotesingle{}}\NormalTok{)}
\end{Highlighting}
\end{Shaded}

\includegraphics{TS1HW3_files/figure-latex/unnamed-chunk-5-1.pdf}

\hypertarget{ex-3.6}{%
\subsection{Ex 3.6}\label{ex-3.6}}

\begin{Shaded}
\begin{Highlighting}[]
\FunctionTok{data}\NormalTok{(beersales)}
\FunctionTok{plot}\NormalTok{(beersales)}
\end{Highlighting}
\end{Shaded}

\includegraphics{TS1HW3_files/figure-latex/unnamed-chunk-6-1.pdf}

\begin{Shaded}
\begin{Highlighting}[]
\FunctionTok{plot}\NormalTok{(beersales, }\AttributeTok{ylab=}\StringTok{\textquotesingle{}Monthly Beer Sales\textquotesingle{}}\NormalTok{, }\AttributeTok{type=}\StringTok{\textquotesingle{}l\textquotesingle{}}\NormalTok{)}
\FunctionTok{points}\NormalTok{(}\AttributeTok{y=}\NormalTok{beersales,}\AttributeTok{x=}\FunctionTok{as.vector}\NormalTok{(}\FunctionTok{time}\NormalTok{(beersales)), }\AttributeTok{pch=}\FunctionTok{as.vector}\NormalTok{(}\FunctionTok{season}\NormalTok{(beersales)))}
\end{Highlighting}
\end{Shaded}

\includegraphics{TS1HW3_files/figure-latex/unnamed-chunk-7-1.pdf}

\begin{Shaded}
\begin{Highlighting}[]
\NormalTok{mnth }\OtherTok{=} \FunctionTok{season}\NormalTok{(beersales)}
\NormalTok{linbeer }\OtherTok{=} \FunctionTok{lm}\NormalTok{(beersales}\SpecialCharTok{\textasciitilde{}}\NormalTok{mnth)}
\FunctionTok{summary}\NormalTok{(linbeer)}
\end{Highlighting}
\end{Shaded}

\begin{verbatim}
## 
## Call:
## lm(formula = beersales ~ mnth)
## 
## Residuals:
##     Min      1Q  Median      3Q     Max 
## -3.5745 -0.4772  0.1759  0.7312  2.1023 
## 
## Coefficients:
##               Estimate Std. Error t value Pr(>|t|)    
## (Intercept)   12.48568    0.26392  47.309  < 2e-16 ***
## mnthFebruary  -0.14259    0.37324  -0.382 0.702879    
## mnthMarch      2.08219    0.37324   5.579 8.77e-08 ***
## mnthApril      2.39760    0.37324   6.424 1.15e-09 ***
## mnthMay        3.59896    0.37324   9.643  < 2e-16 ***
## mnthJune       3.84976    0.37324  10.314  < 2e-16 ***
## mnthJuly       3.76866    0.37324  10.097  < 2e-16 ***
## mnthAugust     3.60877    0.37324   9.669  < 2e-16 ***
## mnthSeptember  1.57282    0.37324   4.214 3.96e-05 ***
## mnthOctober    1.25444    0.37324   3.361 0.000948 ***
## mnthNovember  -0.04797    0.37324  -0.129 0.897881    
## mnthDecember  -0.42309    0.37324  -1.134 0.258487    
## ---
## Signif. codes:  0 '***' 0.001 '**' 0.01 '*' 0.05 '.' 0.1 ' ' 1
## 
## Residual standard error: 1.056 on 180 degrees of freedom
## Multiple R-squared:  0.7103, Adjusted R-squared:  0.6926 
## F-statistic: 40.12 on 11 and 180 DF,  p-value: < 2.2e-16
\end{verbatim}

\begin{Shaded}
\begin{Highlighting}[]
\NormalTok{std.r }\OtherTok{=} \FunctionTok{rstudent}\NormalTok{(linbeer)}
\end{Highlighting}
\end{Shaded}

\begin{Shaded}
\begin{Highlighting}[]
\FunctionTok{plot}\NormalTok{(std.r, }\AttributeTok{x=}\FunctionTok{as.vector}\NormalTok{(}\FunctionTok{time}\NormalTok{(beersales)), }\AttributeTok{ylab =} \StringTok{"Studentized Residuals"}\NormalTok{, }\AttributeTok{xlab =} \StringTok{"Months"}\NormalTok{, }\AttributeTok{type =} \StringTok{\textquotesingle{}l\textquotesingle{}}\NormalTok{)}
\FunctionTok{points}\NormalTok{(}\AttributeTok{y=}\NormalTok{std.r,}\AttributeTok{x=}\FunctionTok{as.vector}\NormalTok{(}\FunctionTok{time}\NormalTok{(beersales)), }\AttributeTok{pch=}\FunctionTok{as.vector}\NormalTok{(}\FunctionTok{season}\NormalTok{(beersales)))}
\end{Highlighting}
\end{Shaded}

\includegraphics{TS1HW3_files/figure-latex/unnamed-chunk-9-1.pdf}

\begin{Shaded}
\begin{Highlighting}[]
\NormalTok{lin2.beer }\OtherTok{=} \FunctionTok{lm}\NormalTok{(beersales}\SpecialCharTok{\textasciitilde{}}\NormalTok{ mnth }\SpecialCharTok{+} \FunctionTok{time}\NormalTok{(beersales) }\SpecialCharTok{+} \FunctionTok{I}\NormalTok{(}\FunctionTok{time}\NormalTok{(beersales)}\SpecialCharTok{\^{}}\DecValTok{2}\NormalTok{))}
\FunctionTok{plot}\NormalTok{(beersales)}
\FunctionTok{lines}\NormalTok{(}\AttributeTok{y=}\NormalTok{lin2.beer}\SpecialCharTok{$}\NormalTok{fitted.values, }\AttributeTok{x =} \FunctionTok{as.vector}\NormalTok{(}\FunctionTok{time}\NormalTok{(beersales)), }\AttributeTok{col=}\StringTok{\textquotesingle{}red\textquotesingle{}}\NormalTok{)}
\end{Highlighting}
\end{Shaded}

\includegraphics{TS1HW3_files/figure-latex/unnamed-chunk-10-1.pdf}

\begin{Shaded}
\begin{Highlighting}[]
\FunctionTok{plot}\NormalTok{(}\AttributeTok{y=}\FunctionTok{rstudent}\NormalTok{(lin2.beer), }\AttributeTok{x=}\FunctionTok{as.vector}\NormalTok{(}\FunctionTok{time}\NormalTok{(beersales)), }\AttributeTok{ylab =} \StringTok{"Studentized Residuals"}\NormalTok{, }\AttributeTok{xlab=}\StringTok{"Time"}\NormalTok{, }\AttributeTok{type=}\StringTok{\textquotesingle{}l\textquotesingle{}}\NormalTok{)}
\FunctionTok{points}\NormalTok{(}\AttributeTok{y=}\FunctionTok{rstudent}\NormalTok{(lin2.beer), }\AttributeTok{x=}\FunctionTok{as.vector}\NormalTok{(}\FunctionTok{time}\NormalTok{(beersales)), }\AttributeTok{pch=}\FunctionTok{as.vector}\NormalTok{(mnth))}
\end{Highlighting}
\end{Shaded}

\includegraphics{TS1HW3_files/figure-latex/unnamed-chunk-11-1.pdf}

\hypertarget{ex-3.11}{%
\subsection{Ex 3.11}\label{ex-3.11}}

\hypertarget{a}{%
\subsubsection{(a)}\label{a}}

\begin{Shaded}
\begin{Highlighting}[]
\NormalTok{quad.wages }\OtherTok{=} \FunctionTok{lm}\NormalTok{(wages}\SpecialCharTok{\textasciitilde{}}\FunctionTok{time}\NormalTok{(wages) }\SpecialCharTok{+} \FunctionTok{I}\NormalTok{(}\FunctionTok{time}\NormalTok{(wages)}\SpecialCharTok{\^{}}\DecValTok{2}\NormalTok{))}
\FunctionTok{summary}\NormalTok{(quad.wages)}
\end{Highlighting}
\end{Shaded}

\begin{verbatim}
## 
## Call:
## lm(formula = wages ~ time(wages) + I(time(wages)^2))
## 
## Residuals:
##       Min        1Q    Median        3Q       Max 
## -0.148318 -0.041440  0.001563  0.050089  0.139839 
## 
## Coefficients:
##                    Estimate Std. Error t value Pr(>|t|)    
## (Intercept)      -8.495e+04  1.019e+04  -8.336 4.87e-12 ***
## time(wages)       8.534e+01  1.027e+01   8.309 5.44e-12 ***
## I(time(wages)^2) -2.143e-02  2.588e-03  -8.282 6.10e-12 ***
## ---
## Signif. codes:  0 '***' 0.001 '**' 0.01 '*' 0.05 '.' 0.1 ' ' 1
## 
## Residual standard error: 0.05889 on 69 degrees of freedom
## Multiple R-squared:  0.9864, Adjusted R-squared:  0.986 
## F-statistic:  2494 on 2 and 69 DF,  p-value: < 2.2e-16
\end{verbatim}

\hypertarget{b}{%
\paragraph{(b)}\label{b}}

\begin{Shaded}
\begin{Highlighting}[]
\FunctionTok{runs}\NormalTok{(}\FunctionTok{rstudent}\NormalTok{(quad.wages))}
\end{Highlighting}
\end{Shaded}

\begin{verbatim}
## $pvalue
## [1] 1.56e-07
## 
## $observed.runs
## [1] 15
## 
## $expected.runs
## [1] 36.75
## 
## $n1
## [1] 33
## 
## $n2
## [1] 39
## 
## $k
## [1] 0
\end{verbatim}

Very low p-value suggests we reject the independence of residuals.

\hypertarget{c}{%
\paragraph{(c)}\label{c}}

\begin{Shaded}
\begin{Highlighting}[]
\FunctionTok{acf}\NormalTok{(}\FunctionTok{rstudent}\NormalTok{(quad.wages))}
\end{Highlighting}
\end{Shaded}

\includegraphics{TS1HW3_files/figure-latex/unnamed-chunk-14-1.pdf}

High AutoCorrelation

\hypertarget{d}{%
\paragraph{(d)}\label{d}}

\begin{Shaded}
\begin{Highlighting}[]
\FunctionTok{hist}\NormalTok{(}\FunctionTok{rstudent}\NormalTok{(quad.wages))}
\end{Highlighting}
\end{Shaded}

\includegraphics{TS1HW3_files/figure-latex/unnamed-chunk-15-1.pdf}

\begin{Shaded}
\begin{Highlighting}[]
\FunctionTok{qqnorm}\NormalTok{(}\FunctionTok{rstudent}\NormalTok{(quad.wages))}
\FunctionTok{qqline}\NormalTok{(}\FunctionTok{rstudent}\NormalTok{(quad.wages))}
\end{Highlighting}
\end{Shaded}

\includegraphics{TS1HW3_files/figure-latex/unnamed-chunk-15-2.pdf}

SKEWED

\end{document}
