\documentclass[a5paper,10pt]{article}

\usepackage{amsmath}
\usepackage{mathtools, cancel}

\usepackage{xepersian}
\usepackage{fontspec}
\settextfont{B Mitra}
\setlatintextfont[Scale=1]{TeX Gyre Termes}
\usepackage{fullpage}
\usepackage{setspace}
\doublespacing

\author{پوریا عصاره ها}
\title{
    تمرین کلاسی 30 آبان\\
    سری زمانی ۱ \\
    دکتر افتخاری
}
\date{\today}
\begin{document}
\maketitle
\[ \forall k > q  Var(r_k) \approx \frac{1}{n} \left[ 1 + \sum_j^q \rho^2_j \right] \]

به ترتیب برای مرتبه \lr{MA} آزمون میکنیم.


\[ H_0: \rho_1 = 0 \quad \Rightarrow \quad r_1 = -0.49 \quad , \quad Var(r_1) = \frac{1}{n} \] 
\[ r_1 \ \notin \ (- 1.96\sqrt{\frac{Var(r_1)}{n}}, +1.96\sqrt{\frac{Var(r_1)}{n}}) = (-0.0196, 0.0196) \]

فرض خنثی برای $\rho_1$ رد می شود. در نتیجه صفر نیست و $q=1$ را می توان پذیرفت.

حال سراغ مرتبه بعدی می رویم.

\[ H_0: \rho_2 = 0 \quad \Rightarrow \quad r_2 = 0.31 \quad , \quad Var(r_2) = \frac{1}{n}\left[1 + 2\rho_1^2\right] \] 

مقدار $\rho_1$ معلوم نیست اما میدانیم وجود دارد. برآوردش را جایگزین می کنیم تا برآورد واریانس $r_2$ بدست آید.

\[ \hat{Var}(r_2) = \frac{1}{n}\left[1 + 2r_1^2 \right] = \frac{1.4802}{100} \Rightarrow \quad 0 \notin ( 0.07 , 0.55) \]

فرض خنثی برای $\rho_2$ رد می شود. در نتیجه صفر نیست و $q=2$ را می توان پذیرفت.

حال سراغ مرتبه بعدی می رویم.

\[H_0: rho_3 = 0 \quad \Rightarrow \quad r_3 = - 0.21 \quad , \quad Var(r_3) = \frac{1}{n}\left[1 + 2\rho_1^2 + 2\rho_2^2\right] \]

همانند قبلی از برآورد واریانس استفاده می کنیم.

\[ Var(r_3) =  0.0167 \quad \Rightarrow \quad 0 \notin (- 0.45, + 0.04) \]

$\rho_3 = 0$ رد نمی شود. 

بنابراین ماکسیمم $q$ برای این داده را ۲ در نظر میگیریم. 

\lr{MA(2)}

برای لگ های زیاد از تقریب یک صدم برای واریانس استفاده می کنیم و همگی صفر بودنشان رد نمی شود. 

و برای قستمت اتو رگرسیو باید $\phi_{kk}$ها محاسبه و برآورد شوند... 

و خیلی طاقت فرسا خواهد بود.
\end{document}